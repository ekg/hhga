\documentclass[11pt, a4paper]{article}
\usepackage{nips_2016}
\usepackage{algorithm,algpseudocode,amsfonts,xcolor}

\title{} %https://scholar.google.cz/scholar?q=genotype&btnG=&hl=en&as_publication=neural+information+processing+systems&as_sdt=0%2C5&as_ylo=2012
\author{NDP,EG, PD?}
\date{\today}

\begin{document}
\maketitle

% the closest papers to ours are: http://papers.nips.cc/paper/4782-scalable-imputation-of-genetic-data-with-a-discrete-fragmentation-coagulation-process



\section{Introduction}

We present a state of the art genotype caller that learns to combine the strength of different underlying calling strategies (freebayes, fermkit).
TODO:Explain why they have different strenghts (for general audience) 
These alone achieve performance ocmpeitive with the state of the art in a recent FDA sponsored challenge.
We then show how to leverage data obtained from thousands of other genomes using different sequencing techniques. 
Thisexpands the potential o mahcine learning systems in tasks where the underlying technical inovation drving the dta generation process. 

TODO: Describe the state of the art, GATK best practices, limitations (beyond libre software) which we overcome.


Describe 1000G data  

\section{Pipeline/ Bio}

describe how we create the hhga files

\section{Theory}

A subject (each subject is  a few million (N) classification tasks) for whom we have  x and  need to predict y.
Another subject for whom have x,  x*,  y and  y*.
Thousand subjects for whom only x* and y*.

The star variables are lower quality but much more abudnant.
Explain quality as mainly driven by depth of reads.
Assumption mulitple samples from the low quality can encode the full information (or close to) in the high quality.

Approach one: learn in the low quality space x* to y*, use the subject for whom we have both star and non star to learn to transform one sample of non-star into K samples of star, so that we mantain perforance on the prediction task relative to training directly on star.


\section{Empirical Performance}

A table with the various flags

\section{Conclusions Further}

theres a lot of implicit parameters and functional  that are ad hoc picked in the pipeline, can we optimize over those? 


can we learn accross species, accross machines? more explcit hierarchical modeling? structured loss funcitons? 

